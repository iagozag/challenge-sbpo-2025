\documentclass[12pt]{article}

\usepackage[a4paper,top=2cm,bottom=2cm,left=3cm,right=3cm,marginparwidth=1.75cm]{geometry}

\usepackage{amsmath}
\usepackage{graphicx}
\usepackage[colorlinks=true, allcolors=blue]{hyperref}

\title{Modelling order picking problem}
\author{Iago Zagnoli Albergaria}

\begin{document}
\maketitle

\section{Conjuntos}

\begin{itemize}
    \item $\mathcal{O}$: Conjunto de pedidos no \textit{backlog}.
    \item $\mathcal{I}_o$: Subconjunto de itens solicitados pelo pedido $o \in \mathcal{O}$.
    \item $\mathcal{I}$: Conjunto de itens, onde $\mathcal{I} = \bigcup_{o \in \mathcal{O}} \mathcal{I}_o$.
    \item $\mathcal{A}_i$: Subconjunto de corredores contendo pelo menos uma unidade do item $i$.
    \item $\mathcal{A}$: Conjunto de corredores, onde $\mathcal{A} = \bigcup_{i \in \mathcal{I}} \mathcal{A}_i$.
\end{itemize}

\section{Constantes}

\begin{itemize}
    \item $u_{oi}$: Número de unidades do item $i \in \mathcal{I}$ solicitado pelo pedido $o \in \mathcal{O}$.
    \item $u_{ai}$: Número de unidades do item $i \in \mathcal{I}$ disponíveis no corredor $a \in \mathcal{A}$.
    \item \textbf{LB / UB}: Limite inferior / superior do \textit{tamanho da wave}.
\end{itemize}


\section{First model}
\subsection{Non-linear}

\begin{align*}
    \max &\quad \frac{\sum_{o \in O} \sum_{i \in I} x_o u_{oi}}{\sum_{a \in A} y_a} \\ \\
    \text{s.t.} \quad & \sum_{o \in O} \sum_{i \in I} x_o u_{oi} \geq LB \\
    & \sum_{o \in O} \sum_{i \in I} x_o u_{oi} \leq UB \\
    & \sum_{o \in O} x_o u_{oi} \leq \sum_{a \in A} y_a u_{ai}, \quad \forall i \in I
\end{align*}

\[x_o =
    \begin{cases}
        1, & \text{if $o \in O$ is in the solution }, \\
        0, &\text{otherwise}
    \end{cases}
\]

\[y_a=
    \begin{cases}
        1, & \text{if $a \in A$ is in the solution }, \\
        0, &\text{otherwise}
    \end{cases}
\]

\subsection{Linearization}

If $w = \frac{1}{\sum_{a \in A}y_a}$, then:
\[
\max &\quad \sum_{o \in O} \sum_{i \in I} x_o w u_{oi}
\]

We know that $w \leq 1$. So, using the big $M$ method, we can linearize the function $x_o w$ by introducing a variable $z_o$ that will be equal to $x_o w$ with some new restrictions.

\begin{itemize}
    \item $z_o \leq w, \forall o \in O$.
    \item $z_o \leq x_o, \forall o \in O$.
    \item $z_o \geq w + (x_o-1), \forall o \in O$.
\end{itemize}

Now, we have:

\begin{align*}
    \max &\quad \sum_{o \in O} \sum_{i \in I} z_o u_{oi} \\ \\
    \text{s.t.} \quad & \sum_{o \in O} \sum_{i \in I} x_o u_{oi} \geq LB \\
    & \sum_{o \in O} \sum_{i \in I} x_o u_{oi} \leq UB \\
    & \sum_{o \in O} x_o u_{oi} \leq \sum_{a \in A} y_a u_{ai}, \quad \forall i \in I \\
    & w = \frac{1}{\sum_{a \in A}y_a} \\
    & z_o \leq w, \forall o \in O \\
    & z_o \leq x_o, \forall o \in O \\
    & z_o \geq w + (x_o-1), \forall o \in O \\
    & z_o \geq 0, \forall o \in O \\ \\
    & x_o \in B, \forall o \in O \\
    & y_a \in B, \forall a \in A
\end{align*}

\end{document}
